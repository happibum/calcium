\documentclass{report}
\usepackage{amsmath}

\begin{document}
\title{Analysis Specification of YAtern}
\author{Se-Won Kim}
\maketitle

The analysis is basically a pointer analysis.

\chapter{Abstraction of Values}
\section{Abstract Values}
An abstract value for an expression is defined as follows.
\[
A = N \times S \times B \times O.
\]
The value of an expression at a specific moment could be a primitive number, 
a primitive string, a primitive boolean or an object.
We collect those values according to some criteria 
and abstract into an element of $A$.
Therefore, we use the 4-ary tuples:
$N$ is the abstract domain for number values, 
$S$ for string values, and $O$ for object values. 

Note that we do not have an abstraction for the primitive 
value \texttt{undefined}. 
In some sense, any abstraction in $A$ implicitly include 
the concrete value \texttt{undefined}.
This decision is natural because of the following two reasons.
\begin{itemize}
\item Since the analysis is flow-insensitive,
  any variable could be \texttt{undefined} before its flow-insensitive use.
\item It is a burden to explicitly represent them in the abstract values.
\end{itemize}
We ignore \texttt{null} value, which is of object type,
as we ignore \texttt{undefined} 
since \texttt{null} is often interchangable 
with \texttt{undefined} in practice.

\section{Primitive Types}
We can parameterize the abstract domains 
\[
N = \{ \}
\]

\section{Object Types}

\section{Function Types}


\chapter{Constraint Generation}

\chapter{Basic Constraint Solver}

In this chapter, we show the overall design of the constraint solver. 

One of the most important principles of the solver is that 
we ignore the possibility of \texttt{null} and \texttt{undefined} 
during constraint solving.
Usually using those void values is not the intension of the programmer.
If we take those values into account 
we have much more spurious abstract values.
For example, consider \texttt{c = a + b} where \texttt{a} and \texttt{b} 
are always string values. 
However, the abstractions of the variables contain \texttt{undefined} and
\texttt{undefined + undefined === NaN}, which is of number type. 


\chapter{Context Dependent Analysis}
\begin{enumerate}
\item extend-like functions
\item special case of for-in loops
\end{enumerate}

\chapter{Optimizing the Solver: Forwarding Types}

\end{document}